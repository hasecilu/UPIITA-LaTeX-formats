\section{Modos de operaci\'on}

Para dise\~{n}ar el SM se deben de tener en cuenta los distintos modos que deber\'a soportar de modo que en la siguiente tabla se presentan los modos de trabajo de la m\'aquina Pick \& Place.
\newline
Estos modos de operaci\'on est\'an basados en m\'aquinas comerciales por lo que ser\'a com\'un encontrar sistemas con funcionalidades parecidas.
\newline

\begin{center}
\begin{tabular}{ | m{2cm} | m{5cm} | }
 \hline
Modos de operaci\'on  & Descripci\'on \\
 \hline\hline
 Modo autom\'atico & En este modo el robot es capaz de ejecutar las instrucciones en tiempo real, es decir, recibir\'a en vivo las instrucciones en c\'odigo G, del sistema inform\'atico (computadora externa),  que controlar\'an el movimiento del efector final, cabe mencionar que las trayectorias ya est\'an calculadas para realizar alg\'un trabajo en concreto.\\
\hline
 Modo semiautom\'atico & En este modo el robot es capaz de moverse de acuerdo a comandos mandados por el usuario, es decir, recibir\'a en vivo las instrucciones en c\'odigo G, a voluntad del operario,  que controlar\'an el movimiento del efector final.\\
\hline
 Modo trabajo continuo & En este modo el robot es capaz de moverse de acuerdo a comandos guardados en memoria, es decir, leer\'a las instrucciones en c\'odigo G desde un dispositivo de memoria lo que permitir\'a funcionar sin conexi\'on a una computadora externa.\\
\hline
 Modo de diagn\'ostico & Este modo permitir\'a al robot detectar malfuncionamientos que puedan interferir con el buen funcionamiento del robot, como test de repetibilidad, test de presici\'on entre otros.\\
\hline
 Modo de calibraci\'on & Este modo permitir\'a al robot a disminuir al m\'inimo las desviaciones que pudieran haber saltado a la luz despu\'es de realizar un diagn\'ostico.\\
\hline
\end{tabular}
\end{center}
