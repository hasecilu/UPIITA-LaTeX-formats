\section{Propuesta de solución}
\label{Propuesta de solución}
1 200 − 2 400 palabras\\

\subsection{Definición de la metodología mecatrónica}
Escoger alguna de las siguientes
% To do list: add references
\begin{itemize}
	\item VDI-Standard: VDI-2206 es un estándar alemán para el diseño de sistemas mecatrónicos
	\item De Silva presenta en Mechatronic systems: devices, design, control, operation and monitoring. conceptos generales de sistemas mecatrónicos
	\item Shetty y Kolk en Mechatronics system design, SI version desarrollan un modelo para diseñar sistemas mecatrónicos
\end{itemize}

\subsection{Desarrollo de la propuesta de solución}
\lipsum[15]

Las necesidades son especificaciones que da el usuario y se deben de cubrir de forma satisfactoria, mientras
que los requerimientos se definen como especificaciones puntuales de un sistema, que incluyen una
declaración de valor y unidades.\\

\begin{table}[!ht]
	\centering
	\caption{Tabla de necesidades}
	\begin{tabular}{ccc}
		\toprule
		\textbf{No.} & \textbf{Necesidad} & \textbf{Clasificación} \\ \midrule
		N\textsubscript{1} &   Ser eficiente    &       Funcional        \\
		N\textsubscript{2} &   Ser eficiente    &      No funcional      \\ \bottomrule
	\end{tabular}
\end{table}

Se convierten las necesidades en requerimientos medibles
\begin{table}[!ht]
	\centering
	\caption{Tabla de requerimientos}
	\begin{tabular}{cccc}
		\toprule
		\textbf{No.} & \textbf{Requerimiento} & \textbf{Variable} & \textbf{Valor} \\ \midrule
	 R\textsubscript{1}   &       Velocidad        &       $ v $       &    0 [m/s]     \\
	 R\textsubscript{2}   &          Peso          &       $ w $       &     20 [N]     \\ \bottomrule
	\end{tabular}
\end{table}

\textbf{{\large Funciones que componen al sistema}}

FBS (Functional Breakdown Structure)

\textbf{Función principal}: asd\\

F\textsubscript{1}: Función relacionada con el propósito del sistema. 
\begin{itemize}
	\item El sistema debe de ...
\end{itemize}

F\textsubscript{2}: Gestionar energía. 
\begin{itemize}
	\item El sistema debe de transformar la energía de entrada 
	\item El sistema debe de suministrar la energía a todos los subsistemas que la requieran
\end{itemize}

F\textsubscript{3}: Gestionar información. 
\begin{itemize}
	\item El sistema debe de medir los parámetros de interés
	\item El sistema debe de acondicionar las señales de las variables de interés
	\item El sistema debe de procesar la información
	\item El sistema debe de almancenar la información para ...
	\item El sistema debe de tomar desiciones
	\item El sistema debe de comunicar a los diferentes subsistemas
\end{itemize}

F\textsubscript{4}: Soportar y proteger al sistema. 
\begin{itemize}
	\item El sistema debe de soportar todas sus partes y mantenerse rígido %( que no esté guango XD)
	\item El sistema debe de proteger las partes críticas pra evitar su destrucción
	\item El sistema debe de proteger al usuario del sistema % must protect that smile!
\end{itemize}

F\textsubscript{5}: Gestionar el modo de operación del sistema. 
\begin{itemize}
	\item El sistema debe gestionar el comportamiento del sistema en los modos de operación: manual,
	automático, semiautomático, diagnóstico, de calibración y de espera. % Ejemplos
\end{itemize}

F\textsubscript{6}: Interactuar con el usuario. 
\begin{itemize}
	\item El usuario deberá poder escoger entre los modos de operación
	\item El usuario podrá visualizar el estado del sistema por medio de una interfaz% aplicación móvil, etc.
\end{itemize}
	
% https://tex.stackexchange.com/a/299500/
% Ejemplo de FBS, no coincide con lo anterior
\begin{center}
	\resizebox{0.9\textwidth}{!}{%
		\begin{forest}
			for tree={
				% forked edges,
				draw,
				rounded corners,
				node options={align=center,},
				text width=2.7cm,
				minimum width=2cm, minimum height=1cm
			},
			where level=0{%
			}{%
				folder,
				grow'=0,
				if level=1{%
					before typesetting nodes={child anchor=north},
					edge path'={(!u.parent anchor) -- ++(0,-25pt) -| (.child anchor)},
				}{},
			}
			[Función principal, parent
				[F\textsubscript{1}, child
					[F\textsubscript{1.1}, grandchild
						[F1.1.1, for tree={fill=colorIPN!10}]
						[F1.1.2, for tree={fill=colorIPN!10}]
					]
					[F1.2, grandchild]
					[F1.3, grandchild
						[F1.3.1, for tree={fill=colorIPN!10}]
						[F1.3.2, for tree={fill=colorIPN!10}]
						[F1.3.2, for tree={fill=colorIPN!10}]
					]
					[F1.4, grandchild]
					[F1.5, grandchild]
				]
				[F\textsubscript{2}, child
					[F2.1, grandchild]
					[F2.2, grandchild]
					[F2.3, grandchild]
					[F2.4, grandchild]
					[F2.5, grandchild]
					[F2.6, grandchild]
					[F2.7, grandchild]
				]
				[F\textsubscript{3}, child
					[F3.1, grandchild]
					[F3.2, grandchild]
					[F3.3, grandchild]
					[F3.4, grandchild
						[F3.4.1, for tree={fill=colorIPN!10}]
						[F3.4.2, for tree={fill=colorIPN!10}]
					]
				]
				[F\textsubscript{4}, child
					[F4.1, grandchild]
					[F4.2, grandchild]
					[F4.3, grandchild]
					[F4.4, grandchild
						[F4.4.1, for tree={fill=colorIPN!10}]
						[F4.4.2, for tree={fill=colorIPN!10}]
					]
				]
				[F\textsubscript{5}, child
					[F5.1, grandchild]
					[F5.2, grandchild]
					[F5.3, grandchild
						[F5.3.1, for tree={fill=colorIPN!10}]
						[F5.3.2, for tree={fill=colorIPN!10}]
						[F5.3.3, for tree={fill=colorIPN!10}]
					]
					[F5.4, grandchild]
				]
				[F\textsubscript{6}, child
					[F6.1, grandchild]
					[F6.2, grandchild]
					[F6.3, grandchild]
					[F\textsubscript{6.4}, grandchild
						[F\textsubscript{6.4.1}, for tree={fill=colorIPN!10}]
						[F\textsubscript{6.4.2}, for tree={fill=colorIPN!10}]
					]
					[F6.5, grandchild]
				]
			]
		\end{forest}
	}
\end{center}

% Ejemplo de integración de módulos y subssitemas, no coincide con lo anterior
\begin{tabular}{c p{150mm}}
	S\textsubscript{1} & \textit{Sistema estructural}: Se encarga de \underline{integrar} a todos los componentes que conforman el sistema así como también de protegerlos de elementos externos y de si mismo.\\
	M\textsubscript{1} & \textit{Módulo energético}: Se encarga de transformar y distribuir la energía eléctrica a todos los componentes del sistema que lo requieren. \\
	M\textsubscript{2} & \textit{Módulo de medición de parámetros}: Se encarga de medir todas aquellas variables significativas para el sistema.\\
	M\textsubscript{3} & \textit{Módulo de movimiento }: Se encarga de trasladar los componentes desde el alimentador de componentes hasta su respectiva posición en la PCB.\\
	M\textsubscript{4} & \textit{Módulo de agarre}: Se encarga de agarrar al componente de su posición original, sostener durante el trayecto y soltar en el lugar adecuado a los componentes. \\
	M\textsubscript{5} & \textit{Módulo de procesamiento de información}: Se encarga de realizar todos los cálculos necesarios para el control, así como de procesar datos de entrada. \\
	M\textsubscript{6} & \textit{Módulo alimentador de componentes}: Se encarga de colocar a los componentes en cierta posición de la cual serán tomados por el módulo de agarre. \\
	M\textsubscript{7} & \textit{Interfaz humano-máquina}: Se encarga de enviar información del sistema al operario  y viceversa, tiene una gran importancia porque a travéz de este módulo funciona el modo de operación semiautomático. \\
	S\textsubscript{2} & \textit{Sistema de procesamiento central}: Es la unión de M2 y M5. \\
	S\textsubscript{3} & \textit{Sistema de manipulación}: Es la unión de M3 y M4.
\end{tabular}

\subsection{Resultados esperados}
Con este prototipo se busca ...

