\documentclass{beamer}

\mode<presentation> {

	% The Beamer class comes with a number of default slide themes
	% which change the colors and layouts of slides. Below this is a list
	% of all the themes, uncomment each in turn to see what they look like.
	
	\definecolor{colorIPN}{RGB}{108 29 69} % Pantone 222 C\\
	\definecolor{darkGray}{RGB}{25 25 25}
	
	\usetheme{Madrid}
	%	Other themes are: AnnArbor, Antibes, Bergen, Berkeley, Berlin, Boadilla, CambridgeUS, Copenhagen, Darmstadt, Dresden, Frankfurt,Goettingen ,Hannover Ilmenau, JuanLesPins, Luebeck, Madrid, Malmoe, Marburg, Montpellier, PaloAlto, Pittsburgh, Rochester, Singapore, Szeged, Warsaw
	
	% As well as themes, the Beamer class has a number of color themes
	% for any slide theme. Uncomment each of these in turn to see how it
	% changes the colors of your current slide theme.
	
	\usecolortheme[named=colorIPN]{structure}
	%	Other color themes are: albatross, beaver, beetle, crane, dolphin, dove, fly, lily, orchid, rose, seagull, seahorse, whale, wolverine
	
%	\setbeamertemplate{footline} % To remove the footer line in all slides uncomment this line
%	\setbeamertemplate{footline}[page number] % To replace the footer line in all slides with a simple slide count uncomment this line
	
%	\setbeamertemplate{navigation symbols}{} % To remove the navigation symbols from the bottom of all slides uncomment this line
}

\usepackage{graphicx} % Allows including images
\usepackage{booktabs} % Allows the use of \toprule, \midrule and \bottomrule in tables








% CONFIGURATION FROM ARTICLETIPOULI.CLS 
%\NeedsTeXFormat{LaTeX2e}
%
%% Name of the class we are creating
%\ProvidesClass{ReportTipoUli}[2020/09/02 Technical Report]
%
%% Now we need to use an existent class
%% I'm using the article class because is very handy
%% and we don't need to use the chapter command to view 
%% the table of contents with no 0.n elements
%\LoadClass[12pt, oneside, a4paper]{article}
%
%%%%%%%%%%%%%%%%%%%%%%%%%%%%%%%%%%% PREAMBLE %%%%%%%%%%%%%%%%%%%%%%%%%%%%%%%%%%%%%%
%
%% If you need to change the margins and related stuff uncomment this section
%%https://en.wikibooks.org/wiki/LaTeX/Page_Layout
%%https://www.overleaf.com/learn/latex/page_size_and_margins
%\RequirePackage{calc}
%\RequirePackage[includeheadfoot]{geometry}
%
%\setlength{\hoffset}{0mm}
%\setlength{\oddsidemargin}{20mm-1in}
%\setlength{\textwidth}{\paperwidth-20mm-15mm}
%\setlength{\marginparsep}{100mm}
%\setlength{\marginparwidth}{0mm}
%
%\setlength{\voffset}{0mm}
%\setlength{\topmargin}{10mm-1in}
%\setlength{\headheight}{12pt}
%\setlength{\headsep}{20mm-10mm-12pt}
%\setlength{\textheight}{\paperheight-20mm-20mm}
%\setlength{\footskip}{10mm}
%
%
%\RequirePackage[main=spanish, english, japanese]{babel}  % Indica que escribiermos en es, en, jp
%%\RequirePackage[utf8]{inputenc}  %Codificación usada: ISO-8859-1(latin1)  o utf8
%%\RequirePackage{fourier}  %logos and ornaments
%\spanishdecimal{.}
%
%
% If you are interested on writing Chinese/Japanese/Korean text you need this package and need to install the HanaMinA and HanaMinB fonts
% https://www.ctan.org/pkg/xecjk
%\RequirePackage{xeCJK}
\usepackage{xeCJK}
%% enable fallback font feature
\xeCJKsetup{AutoFallBack}
\setCJKmainfont{HanaMinA}
% set fallback fonts to `HanaMinA'
\setCJKmainfont[FallBack]{HanaMinB}
%
%
%% Important packages related to maths and important elements.
%
%\RequirePackage{amsmath}  % Comandos extras para matemáticas
%\RequirePackage{amssymb}  % Simbolos matematicos (por lo tanto)
%\RequirePackage{commath}  % Funcionalidades extras para diferenciales, integrales, 
%% etc (\od, \dif, etc)\RequirePackage{graphicx}  % Incluir imágenes en LaTeX
%\RequirePackage{mathtools}  % Comandos extras para matemáticas
%\RequirePackage{mathrsfs}  % Support for using RSFS fonts in maths
%\RequirePackage{cancel}  % Para cancelar expresiones (\cancelto{0}{x})
%
%\RequirePackage{graphicx}  % Enhanced support for graphics
%\RequirePackage{sidecap}  % Para poner el texto de las imágenes al lado
%\RequirePackage{float}  % Improved interface for floating objects
%\RequirePackage{capt-of}  % Permite usar etiquetas fuera de elementos flotantes
%\RequirePackage{wrapfig}  % Text wrapped around figures & tables
%\RequirePackage{subfigure}  % Subfiguras
%
%\RequirePackage{color}  % Foreground and background colour management
%\RequirePackage{xcolor}  % Color extensions for LaTeX and pdfLaTeX 
%\RequirePackage{tcolorbox}  % Coloured boxes, for LaTeX examples and theorems
%\RequirePackage{colortbl} % Add colour to LaTeX tables
%
%\RequirePackage[export]{adjustbox}  % Several macros to adjust boxed content
%\RequirePackage{array} % Extending the array and tabular environments
%\RequirePackage{multicol}  % Intermix single and multiple columns
%\RequirePackage{multirow}  % Intermix single and multiple rows
%
%
%\RequirePackage{enumitem}  % Customize lists
%\RequirePackage{lipsum}  % Dummy text
%\RequirePackage{tikz}
%%\RequirePackage{anysize}  % Para personalizar el ancho de  los márgenes, DO NOT USE, IT GENERATES PROBLEMS WITH GEOMETRY PACKAGE. GEOMETRY > ANYSIZE
%\RequirePackage{fancyhdr} % Control of headers and footers
%
%% Set biblatex as bibliography manager, biblatex > bibtex
\RequirePackage[backend=biber, style=ieee]{biblatex} % Referencias
%
%%\RequirePackage{chemfig} % Draw molecules
%%\RequirePackage{showframe} % Show boxes on document
%
%
%% CUSTOM COMMANDS %%%%%%%%%%%%%%%%%%%%%%%%%%%%%%%%%%%%%%%%%%%%%%%%%%%%
%
%\newtcolorbox{mymathbox}[1][]{colback=white, sharp corners, #1}
%%\begin{mymathbox}[title=Title of the custom box, colframe=colorIPN]
%%	% Text, equations, figures and more
%%\end{mymathbox}
%
%% IPN custom color <3; for hyperlinks
%\definecolor{colorIPN}{RGB}{108 29 69}  % Pantone 222 C
%\definecolor{colorIPNdark}{HTML}{360F22} % Color generated with \usecolortheme[named=colorIPN]{structure} on beamer document
%\definecolor{colorIPNlight}{HTML}{6C1D45} % Color generated with \usecolortheme[named=colorIPN]{structure} on beamer document
%\RequirePackage[colorlinks=true, plainpages=true, citecolor=blue, linkcolor=colorIPNdark]{hyperref}
%
%% Circle for referencing equations alternative, \textcircled is ugly
%\newcommand*\circled[1]{\tikz[baseline=(char.base)]{
%		\node[shape=circle, draw, inner sep=2pt] (char) {#1};}
%}
%
%\RequirePackage{appendix}
%\renewcommand{\appendixname}{Apéndices}
%\renewcommand{\appendixtocname}{Apéndices toc}
%\renewcommand{\appendixpagename}{Apéndices page} 
%
%% Number equations/figures/tables within sections (i.e. 1.1, 1.2, 2.1, 2.2 instead of 1, 2, 3, 4)
%\numberwithin{equation}{section}
%\numberwithin{figure}{section}
%\numberwithin{table}{section}
%
%\renewcommand{\figurename}{Figurín. }
%
%% PAGE FORMAT RELATED %%%%%%%%%%%%%%%%%%%%%%%%%%%%%%%%%%%%%%%%%%%%%%%%%%%%%%%%%
%
%%\marginsize{2cm}{2cm}{2cm}{2cm}  % Izquierda, derecha, arriba, abajo DO NOT USE
%
%\setlength{\parindent}{2em} % Indentation - horizontal length
%\setlength{\parskip}{1em} % Indentation - vertical length
%\renewcommand{\baselinestretch}{1} % Interline space
%% https://www.overleaf.com/learn/latex/paragraph_formatting
%
%% Headers and footers
%%\pagestyle{fancy}
%%\fancyhf{}
%%%\fancyhead[L]{\footnotesize Instituto Polit\'ecnico Nacional}
%%\fancyhead[L]{\footnotesize IPN}  % Left head
%%%\fancyhead[R]{\footnotesize Unidad Profesional Interdisciplinaria en Ingenier\'ia y Tecnolog\'ias Avanzadas}
%%\fancyhead[R]{\footnotesize UPIITA}   % Right head
%%\fancyfoot[R]{\footnotesize Ingenier\'ia Mecatrónica}  % Right foot
%%\fancyfoot[C]{\thepage}  % Center
%%\fancyfoot[L]{\footnotesize Sistema embebido bien ching\'on implementado en FPGA }  % Left foot
%%\renewcommand{\footrulewidth}{0.4pt}
%
%% WTF COMMANDS %%%%%%%%%%%%%%%%%%%%%%%%%%%%%%%%%%%
%
%% (etiquetas de figuras)
%\providecommand{\norm}[1]{\lVert#1\rVert}
%% Permite usar etiquetas fuera de elementos flotantes
%% (etiquetas de figuras)
%\RequirePackage{caption}
%\providecommand{\norm}[1]{\lVer#1\rVert}
%
%\sidecaptionvpos{figure}{c}  % Para que el texto se alinee al centro vertical
%
%
%% EXTRA COMMANDS FOR SOLVING SOME ERRORS %%%%%%%%%%%%%%%%%%%%%%%%%%%%%%%%%%%%%%%
%
%\extrafloats{100}
%% https://tex.stackexchange.com/questions/46512/too-many-unprocessed-floats
%\allowdisplaybreaks
%% https://tex.stackexchange.com/questions/51682/is-it-possible-to-pagebreak-aligned-equations